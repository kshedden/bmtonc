\documentclass[12pt]{article}
\usepackage{parskip}
\usepackage[margin=0.5in]{geometry}
\begin{document}

\noindent{\bf Statistical methods}

We have $n$ observations (dyad-days), each of which is
represented by two 288-dimensional vectors of heart rate
values. The elements of these two vectors correspond to
the 288 blocks of 5 minute duration during a 24-hour day,
and there is one such vector for the patient and one
for the caregiver within each dyad.

Let $\Sigma_{pc} \in {\cal R}^{p\times p}$ denote the patient/caregiver
cross-covariance, and let $\Sigma_{pp} \in {\cal R}^{p\times p}$ and
$\Sigma_{cc} \in {\cal R}^{p\times p}$ denote the patient
and caregiver covariance matrices, respectively.  By analogy with
Canonical Correlation Analysis (CCA), we seek
matrices $A_p, A_C \in {\cal R}^{p\times q}$ that optimize

$$
\frac{|A_p^\prime \Sigma_{pc}A_c|}{[|A_p^\prime \Sigma_{pp}A_p| \cdot |A_c^\prime \Sigma_{cc}A_c|]^{1/2}}
$$

\noindent subject to the constraints $A_p^\prime A_p = I_q$ and
$A_c^\prime A_c = I_q$, where $\|\cdot\|$ is the determinant.

Due to the fairly high dimensionality the optimization posed above tends
to overfit the data and yields results with poor split-sample
reproducibility.  Therefore we combine the above problem with the
PCA-like objective

$$
\frac{|A_p^\prime \Sigma_{pp} A_p|}{|A_p^\prime A_p|}.
$$

Specifically, we combine the CCA and PCA objectives on the log-scale

$$
\lambda_{\rm cca}[\log |A_p^\prime \Sigma_{pc}A_c| - \log |A_p^\prime \Sigma_{pp}A_p| -
\log |A_c^\prime \Sigma_{cc}A_c|] +
\lambda_{\rm pca}[\log |A_p^\prime \Sigma_{
$$

\end{document}
