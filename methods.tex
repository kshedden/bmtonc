\documentclass{beamer}
\setbeamertemplate{footline}[page number]
\usepackage{graphicx}
\usepackage{hyperref}
\hypersetup{colorlinks=true, citecolor=black, urlcolor=black, pdfpagemode=none}
\usepackage{color}

\begin{document}

\begin{frame}

\begin{itemize}

\item Focus here on circadian patterns

\item Estimate mean circadian patterns for caregivers and patients

\item Identify factors that capture the individual deviations from the mean circadian pattern

\item Seek dyadic effects in these individual deviations

\item Seek dyadic effects in the HR traits (persistent, long temporal range, 
repeated over many days) not the
HR states (transient effects with short temporal range, unique to single days)

\end{itemize}

\end{frame}

\begin{frame}

\begin{itemize}

\item $y_{ijdt}$ is the heart rate for dyad $i$, role $j$, day $d$, time $t$; roles are
caregiver ($j=1$) and patient ($j=2$)

\item $\mu_j(t)$ is the mean for role $j$, and $z_{ijdt} = y_{ijdt} - \mu_j(t)$ is the 
role-centered heart rate observation.

\item Use a low-rank decomposition for incomplete matrices (essentially a factor analysis)
to estimate factors $v_k \in {\cal R}^{1440}$ and scores $u_{ijdk}$, so that
$y_{ijdt} \approx \mu_j(t) + u_{ijdk}\cdot v_{kt}$.

\item $k=1, \ldots, K$ factors; use $K=3$ for now (mostly arbitrary choice)

\item Decompose the variance of the scores according to 
$u_{ijdk} = \theta_i + \eta_{ij} + \gamma_{idk} +\epsilon_{ijdk}$, with corresponding
variances $\tau_\theta^2$, $\tau_\eta^2$, $\tau_\gamma^2$, and $\tau_\epsilon^2$.

\item Rotate the factors to maximize $\tau_\theta^2 + 2\tau_\gamma^2$.

\end{itemize}

\end{frame}

\end{document}
